\section{Description of the problem}
	The aim of this problem is to create a database to store temperature measurements using an existing datastructure.

	The tuples are of the form $(t,c)$, with $t$ being the date of the measurement (in the form YYYYMMDDHHmm), and $c$ being the temperature.
	We can also note that we won't have two identical measurements.

	The data structure must have the following basic operation:
	\begin{itemize}
		\item $Add(D, t, c)$ : Add $(t,c)$ to the database $D$.
		\item $Delete(D, t, c)$ : Remove $(t,c)$ of the database $D$.
		\item $Max(D, t)$ : Return the maximum temperature at time $t$.
		\item $Max(D, t1, t2)$ : Return the maximum temperature in $[t1 ; t2]$.
	\end{itemize}
	Each operation should be at worse $\mathcal{O}(\log_2n)$, with a preprocessing time of $\mathcal{O}(n\log_2n)$.

\section{Self-balancing binary search tree}
	Self-balancing binary search tree are great for accessing and modifying elements in $\mathcal{O}(\log_2n)$ time.

	However, it requires a way to sort elements between them. We could sort the elements first by date, and then by temperature.

	Let's first create a basic type for the datapoints:
\begin{lstlisting}
type DP:
	int t
	int c
\end{lstlisting}

	Let's assume we got an base class \verb#SBBST<E>#. We could create the class \verb#TempDB# extending \verb#SBBST<DP>#.
	To have Add and Delete working, we would only need to override the functions \verb#compare(DP d1, DP d2)# and \verb#equal(DP d1, DP d2)#.
\begin{lstlisting}
class TempDB extends SBBST<DP>:
	override bool compare(DP d1, DP d2):
		if d1.t == d2.t:
			return d1.c < d2.c

		return d1.t < d2.t


	override bool equal(DP d1, DP d2):
		return d1.t == d2.t and d1.c == d2.c
\end{lstlisting}

	Now, we can add our two functions $Max$

\begin{lstlisting}
int Max(TempDB D, int t):
	if D == null:
		return -275

	if D.value.t == t:
		return max(D.value.c, 
				   Max(D.right, t))

	if D.value.t < t:
		return Max(D.left, t)

	if D.value.t > t:
		return Max(D.right, t)	

int Max(TempDB D, int t1, int t2):
	if D == null:
		return -275

	if D.value.t in [t1;t2]:
		return max(D.value.c, 
				   Max(D.left, t1, t2), 
				   Max(D.right, t1, t2))

	if D.value.t < t1:
		return Max(D.left, t1, t2)

	if D.value.t > t2:
		return Max(D.right, t1, t2)
\end{lstlisting}
The complexity would be approximately $\mathcal{O}(\log_2n)$.

\section{HashTable}
	Another way of storing the temperatures would be by using a HashTable.
	We could use the date as the key, and the values would lists of the temperatures for this time.

	We can assume that there is a finite number of locations which does the measurements, $m$.
	Thus, the time to add or delete an element would be $\mathcal{O}(1+m)$.

	A way to improve this would be to use binary search trees to store the temperatures. This way, the complexity would become  $\mathcal{O}(1+\log_2m)$.

	Which such a data structure, implementing $Max$ would be really easy to do:
\begin{lstlisting}
int Max(TempDB D, int t):
	% I assume there is a function returning 
	% the maximum value of the tree
	return D.get(t).getMax() 

int Max(TempDB D, int t1, int t2):
	ts = D.keySet().filter(t => t is in [t1; t2])

	return max([D.get(t).getMax() for t in ts])
\end{lstlisting}

	However, the complexity of the second function would no so great. We could improve it with the next hash optimization.

	\subsection{Hash optimization}
		Hash function is always a big question in hashmaps. At the moment, we can use the date as a hash, but with some assumption, we could do far better.

		For instance, we can suppose that the measurements are periodic (one per hour, for example). Let's call the period $p$, in seconds.
		By converting the time $t$ to a timestamp $t'$, we could use a modulo operation.
		Also, we can suppose that there is no measurements before the date $t_0$.

		This way, we could produce the hash function:\[hash(t) = (t' - t_0') \% p .\]

		Now, we could rewrite our second $Max$ function, to use this new property:
\begin{lstlisting}
int Max(TempDB D, int t1, int t2):
	tt1 = to_timestamp(t1)
	tt2 = to_timestamp(t2)

	ts = [tt1 + i*p for i in range(0, (tt2 - tt1)/p)]

	return max([D.get(t).getMax() for t in ts])
\end{lstlisting}

\section{Conclusion}
	I found different approach for this problem. The first one works only be using the problem informations, while the second one need some assumptions (probably close to reality).